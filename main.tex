\documentclass[12pt,a4paper,ngerman,captions=tableheading]{scrartcl}

% -----------------------------------------------
% --------PAKETE-LADEN---------------------------
% -----------------------------------------------
\usepackage{rotating}
\usepackage{acronym}
\usepackage{graphicx}
\usepackage{longtable}
\usepackage{booktabs}


% -----------------------------------------------
% --------Seitengeometrie------------------------
% -----------------------------------------------

\usepackage{geometry}
\geometry{a4paper, top=27mm, left=30mm, right=20mm, bottom=35mm, headsep=10mm, footskip=12mm}

% -----------------------------------------------
% --------Schrift-+-Wörterbuch-------------------
% -----------------------------------------------

\usepackage[T1]{fontenc}
\usepackage[utf8]{inputenc}
\usepackage{babel}
\usepackage{lmodern}

\usepackage{microtype}
    % \setlength{\emergencystretch}{1em}

\usepackage[scaled]{helvet}
    % Serifenfreie Schrift: 
    % \renewcommand{\familydefault}{\sfdefault}

    % Zeilenabsatz - Einrücken verhindern:
\setlength{\parindent}{0em} 

    % Zeilenabstand auf 1,5: 
\usepackage[onehalfspacing]{setspace}


%------------------------------------------------------
%-----KOPF-und-FUSSZEILEN------------------------------
%------------------------------------------------------

\usepackage{scrlayer-scrpage}
%\usepackage[headtopline,headsepline]{scrlayer-scrpage}
%\setheadtopline{0,4pt}
%\setheadsepline{}

\clearpairofpagestyles
\ihead{ Spezielle Messtechnik (MST2) }
\chead{SAM}
\ohead{\today}
\ifoot*{}
\ofoot{\pagemark}
\pagestyle{scrheadings}
\setkomafont{pageheadfoot}{\small}
\setkomafont{pagehead}{}
\usepackage{breakurl}


%------------------------------------------------------
%--------Physik und Mathe------------------------------
%------------------------------------------------------

\usepackage{upgreek}

\usepackage{chemformula}


\usepackage{chemmacros}

    

\usepackage{mathtools} % lädt auch amsmath

\usepackage{amssymb}
    % & vor entsprechendem Zeichen richtet nachfolgende Formeln daran aus
    % Leerzeilen beenden den Modus! 

\usepackage{siunitx}
\sisetup{inter-unit-product = \cdot}
\sisetup{detect-family}

    % detect-weight, % Schrifttyp übernehmen (bold, kursiv etc) 
    % detect-family, %Schriftart des Umgebungstextes übernehmen (mit/ohne Serifen) 
\DeclareUnicodeCharacter{2212}{-}

% -----------------------------------------
% ---------Links--------------------------------
% -----------------------------------------

\usepackage{url}
\urlstyle{same}



% -----------------------------------------
% -------Listen --------------------------------
% -----------------------------------------

\usepackage{mdwlist}
\usepackage{paralist}

% -------------------------------------------------------------
% Tabellen und Grafiken--------------------------------
% -------------------------------------------------------------
\usepackage{makecell}
\usepackage{graphicx}
\usepackage[lofdepth,lotdepth]{subfig}
%\usepackage{subfigure}
\usepackage{pdfpages}
\usepackage{float}
\usepackage{pdflscape}
\usepackage{tabularx}
\newcolumntype{L}[1]{>{\raggedright\arraybackslash}m{#1}} % linksbündig mit Breitenangabe
\newcolumntype{C}[1]{>{\centering\arraybackslash}m{#1}} % zentriert mit Breitenangabe
\newcolumntype{R}[1]{>{\raggedleft\arraybackslash}m{#1}} % rechtsbündig mit Breitenangabe

\usepackage{multirow}
\usepackage{array}
\usepackage{colortbl}
\definecolor{MKHgreen}{RGB}{65, 160, 55}
\usepackage{tikz}
\usetikzlibrary{external}
%\tikzexternalize
\usepackage{adjustbox}
\usepackage{pgfplots}
\usepgfplotslibrary{colorbrewer}
\pgfplotsset{compat = newest} 
% Tikz is loaded automatically by pgfplots
\usetikzlibrary{pgfplots.statistics, pgfplots.colorbrewer} 
% provides \pgfplotstabletranspose
\usepackage{pgfplotstable}
\usepackage{filecontents}

\usepackage{framed}
\usepackage{xcolor}
\colorlet{shadecolor}{gray!25}
    %\begin{shaded}
    %\end{shaded}



    %Bildunterschriften // justification=raggedright längere Bildunterschriften/Tabellenüberschriften werden linksbündig ausgerichtet, singlelinecheck=false auch kurze Bildunterschriften/Tabellenüberschriften werden linksbündig ausgerichtet
    
\usepackage{caption}
\usepackage{subcaption}
%\usepackage[labelfont=bf,font={small,sf},justification=raggedright,singlelinecheck=false]{caption} 
\addto\captionsngerman{\renewcommand{\figurename}{Abb.}}
\addto\captionsngerman{\renewcommand{\tablename}{Tab.}}


%------------------------------------------------------
%-----HYPERREF-----------------------------------------
%------------------------------------------------------

\usepackage[
  colorlinks, 
  backref=true, 
  pdfpagelabels, 
  pdfstartview=FitH, 
  bookmarksopen=true, 
  bookmarksnumbered=true, 
  pdftitle={Berichte}, 
  pdfauthor={Ahmed EN-NOUR}, 
  pdfsubject={MST}, 
  pdfcreator={}, 
  pdfproducer={}, 
  pdfkeywords={MST, Messung, Labor}, 
  pdfnewwindow=true, 
  linkcolor=black, 
  plainpages=false, 
  hypertexnames=false, 
  citecolor=black, 
  filecolor=black, 
  urlcolor=black
]{hyperref}


    % Links auf Gleitumgebungen springen nicht zur Beschriftung, sondern zum Anfang der Gleitumgebung:

\usepackage[figure,table]{hypcap}

    % \usepackage[table]{hypcap}
    % oder  
    % \usepackage[tabularx]{hypcap}
    

\usepackage[labelfont=bf,justification=RaggedRight,textfont=it]{caption}
\DeclareCaptionLabelSeparator{period-newline}{. \newline}

%------------------------------------------------------
%-----LITERATURQUELLEN---------------------------------
%------------------------------------------------------

\usepackage[babel, german=quotes]{csquotes}
\usepackage[
backend=biber,
style=numeric,
sorting=none
]{biblatex}
\addbibresource{quellen.bib}


\usepackage{wrapfig}
\graphicspath{{./images/}}
\bibliography{Einzelnachweise}
    % Passt die Schriftgröße  im Quellenverzeichnis an die Normale Schriftgröße an:
\renewcommand*{\bibfont}{\normalfont}


\sisetup{output-decimal-marker = {,}}
\pgfkeys{/pgf/number format/use comma}


%------------------------------------------------------
%-----NUMMERIERUNG-------------------------------------
%------------------------------------------------------

\setcounter{secnumdepth}{3}
\usepackage{multirow}
%------------------------------------------------------
%-----WORTTRENNUNG-------------------------------------
%------------------------------------------------------

\usepackage{pgfplots}
\pgfplotsset{width=10cm,compat=1.9,every mark/.append style={solid},}
%\pgfplotsset{every mark/.append style={solid},}


% We will externalize the figures
\usepgfplotslibrary{external}

\hyphenation{Tren-nungs-kor-rek-tur}


%\usepackage[Labor=true,backref=true]{hyperref}
%------------------------------------------------------
%------------------DOKUMENT----------------------------
%------------------------------------------------------

\begin{document}

%------------------------------------------------------
%------------------Titelseite--------------------------
%------------------------------------------------------

\begin{titlepage}
\centering
\begin{onehalfspace}
    \huge \textbf{}  
    \linebreak \large \textbf{}
\end{onehalfspace}

%------------------------------------------------------
%------------------------Logo--------------------------
%------------------------------------------------------

\includegraphics[height=70pt]{Bilder/fhkiel_logo.jpg}

\vspace{0.6cm}

\begin{onehalfspace}
    \huge \textbf{Spezielle Messtechnik (MST2) – Labor}  \linebreak \large \textbf{}
\end{onehalfspace}

\vspace{0.3cm}

\begin{onehalfspace}
    \Large \textbf{Laborversuch V3: SAM}
\end{onehalfspace}

\vspace{1cm}
{\large Teilnehmer:} 
%{\large vorgelegt von:}
    \vspace{0.5cm}

    \begin{tabular}{l l}
        \large\textbf{Alaa Albasha}, & \textbf{Matr. Nr.: 943167} \\
        \large\textbf{Jan-Manuel Megerle}, & \textbf{Matr. Nr.: 942883} \\
        \large\textbf{Nathan Kirori}, & \textbf{Matr. Nr.: 941689} \\
        \large\textbf{Ahmed EN-NOUR}, & \textbf{Matr. Nr.: 937048} \\
    \end{tabular}


    

    \vspace{0.5cm}
    
\large \textbf{MST2\_M2 - Team 1}

\vspace{1cm}
{\large Professor:}

    \vspace{0.2cm}
    
\large \textbf{Prof. Dr.-Ing. Aylin Bicakci }

    \vspace{0.3cm}

\large \textbf{Fachhochschule Kiel \linebreak Sommersemester 2025\\}
{\large Informatik und Elektrotechnik}

\vfill
%\begin{center}
 %       28.10.2023
%\end{center}
\end{titlepage}


\newpage

%------------------------------------------------------
%------------------INHALTSVERZEICHNIS------------------
%------------------------------------------------------

\tableofcontents  
%\textbf{Anhang}
\newpage

%------------------------------------------------------
%------------------INHALT----------------------------
%------------------------------------------------------

\section{Einleitung}
Mikroskope sind in vielen wissenschaftlichen Disziplinen von zentraler Bedeutung und werden regelmäßig eingesetzt. 
Besonders in den Materialwissenschaften spielt das Elektronenmikroskop eine wichtige Rolle, da es die Analyse der topologischen Struktur von Materialien sowie die Identifikation ihrer Bestandteile ermöglicht. 
In diesem Versuch wird ein Rasterelektronenmikroskop (REM) zu Ausbildungszwecken verwendet, um einen grundlegenden Einblick in dessen Funktionsweise und Anwendungsmöglichkeiten zu erhalten.      


%------------------------------------------------------
%######################################################
%------------------------------------------------------

\section{Theoretische Grundlagen}
Das Scanning Acoustic Microscope (SAM), auch bekannt als akustisches Mikroskop, wurde 1974 an der Stanford University entwickelt und stellte das erste Instrument dar, das in der Lage war, ein akustisches Bild zu erzeugen.\\

Das Grundprinzip der SAM basiert auf einem Wandler (Transducer), der Ultraschallwellen erzeugt. Diese werden mithilfe einer akustischen Linse fokussiert und auf die zu untersuchende Probe gerichtet. Beim Auftreffen auf das Material werden die Wellen absorbiert, reflektiert oder gestreut, abhängig von den Eigenschaften der inneren Strukturen. Die reflektierten Wellen werden von Detektoren – meist oberhalb oder unterhalb der Probe – erfasst und in elektrische Signale umgewandelt.

Diese elektrischen Signale liefern Informationen über materialinterne Eigenschaften wie Schichtdicke, Dichte und Grenzflächenbeschaffenheit. Die verwendeten Frequenzen der Ultraschallwellen sind dabei entscheidend, da sie die Auflösung und Eindringtiefe bestimmen. Je nach Frequenz können innere Strukturen wie Risse, Delaminationen, Lufteinschlüsse oder Hohlräume detektiert werden.

Eine Vielzahl von Materialien, die in der industriellen Fertigung verwendet werden – etwa Silizium, Epoxidharz, Bonding-Materialien und Metallrahmen – übertragen Ultraschallwellen ausreichend gut, um eine bildgebende Analyse zu ermöglichen.\\

Das am häufigsten genutzte Verfahren der SAM ist der Reflexionsmodus (Pulse-Echo-Verfahren). Dabei wird eine Ultraschallwelle in die Probe eingekoppelt. Ein Teil der Welle wird an inneren Grenzflächen reflektiert und gelangt zurück zum Wandler, wo sie erneut in ein elektrisches Signal umgewandelt wird. Die Intensität und Polarität dieser Signale liefern Rückschlüsse auf die inneren Strukturen des untersuchten Materials.
\\



 \newpage
\vspace{0.2cm}
\begin{figure}
    \centering
    \includegraphics[scale=0.8]{Bilder/samtheorie}
    \caption{Schematische Darstellung des Funktionsprinzips der Scanning Acoustic Microscopy (SAM) im Reflexionsmodus. Die eingestrahlten Ultraschallwellen werden an inneren Strukturen wie Rissen, Grenzflächen, Einschlüsse oder Delaminationen reflektiert und ermöglichen so eine zerstörungsfreie Analyse der Probe.\cite{1}}
    \vspace{0.2cm}
    \label{Abb.1: Schematische Darstellung des Funktionsprinzips der Scanning Acoustic Microscopy (SAM) im Reflexionsmodus. Die eingestrahlten Ultraschallwellen werden an inneren Strukturen wie Rissen, Grenzflächen, Einschlüsse oder Delaminationen reflektiert und ermöglichen so eine zerstörungsfreie Analyse der Probe. }
\end{figure} 
\vspace{0.2cm}
Zur optimalen Übertragung der akustischen Wellen befindet sich die Probe während der Messung in einem Wasserbad. Wasser dient dabei als Kopplungsmedium, da es eine homogene akustische Impedanz aufweist und eine gleichmäßige Grenzfläche zwischen Wandler und Probe bildet.
Die starke Reflexion an der Grenzfläche zwischen Wandler und Luft entsteht, weil die akustische Impedanz von Luft stark von der des Wandlers und der Probe abweicht. Dadurch gelangen fast keine Schallwellen in die Probe, da sie bereits an der Oberfläche reflektiert werden (siehe Abbildung 1).\\
Zusammenfassend betrachtet der Versuch drei zentrale Verbindungstechnologien der Leistungselektronik, die maßgeblich die interne Struktur der Module und somit auch das akustische Reflexionsverhalten im SAM beeinflussen:\\

Das Silbersintern (Ag) ermöglicht eine metallurgische Verbindung zwischen Halbleiterchip und Träger. Aufgrund des hohen Schmelzpunkts und der exzellenten Wärmeleitfähigkeit bietet Silber eine äußerst stabile und thermisch belastbare Kontaktierung [vgl. izm.fraunhofer.de]. Reine Silber-Sinterverbindungen steigern nachweislich die Lebensdauer moderner Leistungsmodule [vgl. izm.fraunhofer.de].\\

Beim Ultraschallschweißen kommt ein hochfrequentes, mechanisches Fügen zum Einsatz, das sich besonders für spröde Materialien wie Keramik eignet. Torsionale Ultraschallpressen ermöglichen beispielsweise die schonende Verbindung von Kupfer-Leadframes mit keramischen Substraten, ohne diese mechanisch zu beschädigen [vgl. telsonic.com].\\

Das Laminieren kombiniert metallische Leiterbahnen mit isolierenden Folien durch Druck und Hitze. Dieser Prozess erfolgt innerhalb definierter Parameter für Temperatur, Druck und Dauer, um eine homogene, blasenfreie Schichtbildung sicherzustellen [vgl. depositonce.tu-berlin.de]. So entstehen stabile, elektrisch isolierende Grenzflächen zwischen Kupferstruktur und Trägermaterial [vgl. depositonce.tu-berlin.de].\\

Durch die gezielte Wahl unterschiedlicher Fokusebenen im SAM lassen sich innerhalb dieser Verbindungsschichten Kontraste und potenzielle Defektstellen sichtbar machen. Die gewonnenen Bilddaten geben somit Aufschluss über die Qualität der Materialanbindung und das strukturelle Verhalten unter realen Belastungsbedingungen.

%------------------------------------------------------
%######################################################
%------------------------------------------------------

%\newpage
\section{Aufgabenstellung}
Ziel dieses Versuchs ist:
Folgende Informationen über die Probe zu ermitteln:

\begin{itemize}
\item Morphologie.
\item Kristallstruktur.
\item Chemische Zusammensetzung.
\item Analyse des EDX Diagramms und Erklärung.
\item Die Untersuchung der Unterschiede zwischen dem SE- und BSE-Detektor.
\end{itemize}

%------------------------------------------------------
%######################################################
%------------------------------------------------------

%

%------------------------------------------------------
%######################################################
%------------------------------------------------------

\section{Proben und Methoden}
Der durchgeführte REM-Versuch untersuchte einen gesinterten Halbleiter mittels SE- und BSE-Detektor.\\
Als Leistungshalbleiter wird ein SIC-Mosfet benutzt, wo drei Strukturen analysiert wurden: \cite{key}
\begin{itemize}
    \item Gate Pad
    \item Sinterpad
    \item Source Pad
\end{itemize}
Zunächst die EDX Analyse einer gesinterten Sinterpaste in einem Al-Tiegel.\\
Für die Analyse wurde folgende Prüfgerät verwendet:
\begin{itemize}
    \item Rasterelektronenmikroskop der Firma Thermo Fisher Scientific
\end{itemize}
Für diese Analyse ist bei beiden Proben ein Rasterelektronenmikroskop zu verwenden. 

\subsection{Phenom XL G2 von Thermo Fisher Scientific}
Das Phenom XL G2 ist ein hochmoderner Rasterelektronenmikroskop, das für präzise Analyse von Strukturen und Zusammensetzung in der Materialwissenschaften Industrie entwickelt wurde:
Das REM besitzt die folgende Eigenschaften: \cite{2}
\begin{itemize}
    \item Automatisierte Abläufe:: Unterstützt die Automatisierung von Routineprozessen, wodurch manuelle und sich wiederholende Arbeitsschritte reduziert werden.
    \item Schneller Analysebeginn: Nach durchschnittlich 60 Sekunden liefert das Gerät erste Bilder.
    \item EDS-Funktion (Elementanalyse): Zur chemischen Analyse der Proben kann das System optional mit einem energie-dispersiven Röntgenspektrometer (EDS) ausgestattet werden, das die Elementzusammensetzung erfasst.
\end{itemize}



%------------------------------------------------------
%######################################################
%------------------------------------------------------
\clearpage
\section{Durchführung}
Die Versuchsdurchführung erfolgt im Gebäude C12, Raum 3.28. Das verwendete Gerät ist der Phenom XL G2.\\
Das Experiment wird in zwei Abschnitten durchgeführt:\\
\begin{enumerate}
\item Untersuchung die Strukturen folgender Teile des SiC-Mosfet (\hyperref[Abb.3: Positionierung der Messpositionen auf SiC-Mosfet]{Abbildung 3}):
    \subitem 1.1. Gate Pad
    \subitem 1.2. Sinterpad
    \subitem 1.3. Source Pad
\item Analyse einer gesinterten Sinterpaste in einem Al-Tiegel mithilfe von EDX Analyse(\hyperref[Abb.4: Probe aus Sinterpaste in Aluminiumtiegel]{Abbildung 4}):
\end{enumerate}
\vspace{0.5cm}
\begin{figure}[H]
    \centering
    \includegraphics[scale=0.95]{Bilder/Screenshot 2025-04-10 185117}
    \caption{Positionierung der Messpositionen auf SiC-Mosfet\cite{key}}
    \vspace{0.2cm}
    \label{Abb.3: Positionierung der Messpositionen auf SiC-Mosfet}
\end{figure} 
\begin{figure}[H]
    \centering
    \includegraphics[scale=0.95]{Bilder/Screenshot 2025-04-10 190248}
    \caption{Probe aus Sinterpaste in Aluminiumtiegel\cite{key}}
    
    \vspace{0.2cm}
    \label{Abb.4: Probe aus Sinterpaste in Aluminiumtiegel}
\end{figure} 
\subsection{Struktur Untersuchung}
Die Untersuchung begann mit dem ersten Teil des MOSFETs, dem Gate-Pad. Dabei wurde – wie bei jedem anderen Teil – das gleiche Verfahren angewendet: 
Zunächst erfolgte die Aufnahme mit dem BSE-Detektor (Backscattered Electrons) bei drei verschiedenen Vergrößerungsstufen:
\begin{enumerate}
    \item 4000x
    \item 10000x
    \item 15000x 
\end{enumerate}
Anschließend wurde die Aufnahme mit dem SE-Detektor (Sekundärelektronen) unter den gleichen Vergrößerungsstufen durchgeführt. 
Daraufhin erfolgte eine Analyse mittels EDX (energie-dispersive Röntgenspektroskopie), um die chemische Zusammensetzung der jeweiligen Strukturen zu bestimmen.\\
Zu guter Letzt wurde die gesinterte Sinterpaste in einem Aluminium-Tiegel mittels EDX analysiert, wobei ihre chemische Zusammensetzung vollständig bestimmt wurde.



%------------------------------------------------------
%######################################################
%------------------------------------------------------

\section{Ergebnisse}
\subsection{Probe 1}

Die erste Probe zeigt eine gleichmäßige, klar abgegrenzte Struktur mit regelmäßig angeordneten quadratischen Elementen (siehe Abbildung~\ref{Abbildung 6 :probe1}). Die Reflexionssignale erscheinen homogen, ohne sichtbare Unterbrechungen oder Unregelmäßigkeiten innerhalb der Materialübergänge. Auffällig ist die hohe Signalklarheit in den bondfreien Zonen. Es lassen sich weder Risse noch Delaminationen erkennen, was auf eine saubere Verarbeitung und intakte Schichten schließen lässt.
\vspace{0.2cm}
\begin{figure}[htbp]
    \centering
    \includegraphics[scale=0.20]{Bilder/Probe11.jpg}
    \caption{C-Scan der Probe 1 mit klar erkennbaren Strukturen und homogener Reflexion.}
    \label{Abbildung 6 :probe1}
\end{figure}
\vspace{0.5cm}
\subsection{Probe 2}
Bei der zweiten Probe wird eine Schicht-für-Schicht-Analyse durchgeführt. In den insgesamt fünf Fokusebenen zeigen sich deutliche Unterschiede in der Signalintensität und den erkennbaren Strukturen. Die obersten Ebenen (Abbildung~\ref{Abbildung 7_1:probe2_1} und \ref{Abbildung 7_2:probe2_2}) erscheinen relativ glatt, wobei eine kreisförmige Auffälligkeit in Form eines dunklen Punktes sichtbar ist. Dies könnte auf eine Lufteinschluss oder lokale Delamination hindeuten.

Mit zunehmender Tiefenlage (Abbildung~\ref{Abbildung 8_1:probe2_3} bis \ref{Abbildung 8_2:probe2_4}) treten vermehrt Leiterstrukturen hervor, die in den oberen Schichten nicht sichtbar waren. Zudem zeigen sich in tieferen Ebenen leichte Schattenbildungen und Signalbrüche entlang einzelner Linien, was auf potenzielle Trennungen oder Materialunregelmäßigkeiten in unteren Bondschichten hinweist.
\vspace{0.2cm}
\begin{figure}[htbp]
    \centering
    \includegraphics[scale=0.30]{Bilder/Probe2_i794_x001.jpg}
    \includegraphics[scale=0.30]{Bilder/Probe2_i794_x002.jpg}
    \caption{Obere Fokusebenen der Probe 2. Die rechte Abbildung zeigt eine auffällige Reflexionsunterbrechung.}
    \label{Abbildung 7_1:probe2_1}
    \label{Abbildung 7_2:probe2_2}
\end{figure}

\begin{figure}[htbp]
    \centering
    \includegraphics[scale=0.30]{Bilder/Probe2_i794_x003.jpg}
    \includegraphics[scale=0.30]{Bilder/Probe2_i794_x004.jpg}
    \caption{Mittlere Fokusebenen der Probe 2 mit zunehmender Sichtbarkeit der Leiterstrukturen.}
    \label{Abbildung 8_1:probe2_3}
    \label{Abbildung 8_2:probe2_4}
\end{figure}
\clearpage
\begin{figure}[htbp]
    \centering
    \includegraphics[scale=0.30]{Bilder/Probe2_i794_x005.jpg}
    \caption{Tiefste gescannte Ebene der Probe 2. Sichtbare Leiterbahnen mit lokalen Unregelmäßigkeiten im unteren Bereich.}
    \label{Abbildung 9:probe2_5}
\end{figure}


\subsection{Probe 3}

In der letzten Probe, einem DoL-Modul, fällt sofort die großflächige, glatte Reflexionsfläche auf (siehe Abbildung~\ref{Abbildung 10:probe3}). Die Struktur wirkt weitgehend homogen. Es sind keine offensichtlichen Risse oder Hohlräume erkennbar. Bemerkenswert ist jedoch die gute Lesbarkeit des Schriftzugs \enquote{FH-KIEL} in der Bildmitte. Dies deutet auf eine hohe Oberflächengüte und gleichmäßige Signalreflexion hin.
\vspace{0.2cm}
\begin{figure}[htbp]
    \centering
    \includegraphics[scale=0.30]{Bilder/Probe3_i795_c.jpg}
    \caption{C-Scan der Probe 3 (DoL-Modul) mit gleichmäßiger Reflexionsfläche und klar erkennbarer Gravurstruktur.}
    \label{Abbildung 10:probe3}
\end{figure}



%------------------------------------------------------
%######################################################
%------------------------------------------------------
\clearpage
\section{Zusammenfassung}
Im Rahmen des Versuchs zur akustischen Mikroskopie wurden drei unterschiedliche Proben untersucht: ein DCB-Modul mit gesinterten Halbleiterchips, ein Kupfer-Leadframe, der mittels Ultraschallschweißung auf ein Keramiksubstrat aufgebracht wurde, sowie ein DoL-Leistungsmodul mit einer organischen Trägerfolie. Ziel der Untersuchung war es, mithilfe des C-Scan-Modus im Reflexionsverfahren (Puls-Echo-Technik) die inneren Schichtstrukturen und Grenzflächen der Materialien zerstörungsfrei darzustellen und zu analysieren. Dabei macht man sich zunutze, dass Ultraschallwellen an Grenzflächen mit unterschiedlicher akustischer Impedanz reflektiert werden, wodurch innere Materialstrukturen sichtbar gemacht werden können \cite{pvateplaSAM, wikiSAM2024}.

Zur Erfassung der inneren Defekte und Heterogenitäten wurden die Proben in mehreren Fokusebenen untersucht also bei variierender Tiefeneinstellung im C-Scan \cite{hennig2025}. Besonders bei Probe 2 wurde dieses Verfahren genutzt, um schichtabhängige Unterschiede im Reflexionsverhalten aufzudecken. Für eine zuverlässige Kopplung der Schallwellen an das Probenmaterial wurde Wasser als Koppelmedium eingesetzt, da Luft aufgrund ihrer geringen akustischen Impedanz kaum Ultraschall überträgt \cite{pvateplaSAM}.

Das verwendete akustische Mikroskop vom Typ KSI V8 ermöglichte hochauflösende, kontrastreiche Aufnahmen der Probenquerschnitte. Die dabei gewonnenen Bilddaten lieferten eine klare Visualisierung möglicher Defekte wie Delaminationen, Inhomogenitäten oder Hohlräume \cite{hennig2025}. Besonders beim DoL-Modul (Probe 3) zeigten sich einzelne kleine Flecken, vermutlich Oberflächenverunreinigungen oder eingeschlossene Luftblasen, konnten detektiert werden. Solche Inhomogenitäten erzeugen aufgrund des stark abweichenden akustischen Widerstandes besonders helle Reflexionssignale, häufig begleitet von einer Phasenumkehr im Echo \cite{wikiSAM2024}.

Besonders hervorzuheben ist die klare Darstellung des Gravurschriftzugs „FH-KIEL“ im C-Scan. Dies deutet auf eine gleichmäßige und intakte Oberflächenstruktur hin, die zu einer starken Reflexion und nur geringer Signalabschwächung führt. In Probe 1 ( DCB-Modul) ließen sich ebenfalls keine strukturellen Schwächen wie Delaminationen oder Risse nachweisen. Die Bondverbindungen erscheinen vollständig intakt, ebenso die metallischen Kontaktflächen, was auf eine homogene Materialanbindung ohne Hohlräume oder Ablösungen schließen lässt.

Im Gegensatz dazu offenbarte Probe 2, also der Kupfer-Leadframe auf Keramik, bei schrittweiser Tiefenfokussierung mehrere lokal begrenzte Auffälligkeiten in den tiefer liegenden Schichten \cite{hennig2025}. Im C-Scan erscheinen diese als deutlich abgegrenzte Zonen mit erhöhter Reflexionsintensität, was auf Schwachstellen in der strukturellen Verbindung hindeutet \cite{pvateplaSAM}. Besonders an der unteren Grenzfläche zwischen Bondschicht und Keramiksubstrat wurden solche Anomalien sichtbar. Es ist anzunehmen, dass es sich hierbei um partielle Delaminationen oder Lufteinschlüsse handelt. Diese verursachen durch den abrupten Wechsel der akustischen Impedanz eine nahezu vollständige Reflexion des Ultraschalls, häufig inklusive Phasenumkehr \cite{wikiSAM2024}.

Durch die gezielte Fokussierung auf diese tieferliegenden Ebenen und das präzise Setzen eines Laufzeitfensters im C-Scan konnte die exakte Tiefenlage dieser Defekte lokalisiert und in der Bildauswertung eindeutig zugeordnet werden \cite{hennig2025}.

%------------------------------------------------------
%######################################################
%------------------------------------------------------

\section{Fazit}
Dieses Experiment hat erfolgreich gezeigt, wie ein Rasterelektronenmikroskop (REM) funktioniert und wie die verschiedenen Detektoren eingesetzt werden. Die Untersuchung des SiC-MOSFETs und der gesinterten Paste lieferte dabei wichtige Einblicke in die Materialeigenschaften und die Leistungsfähigkeit des REM.

Die Ergebnisse belegen, dass die Kombination aus SE-Detektor, BSE-Detektor und EDX-Analyse gut geeignet ist, um Oberflächen zu untersuchen und Materialien zu identifizieren. Durch die EDX-Analyse lassen sich topografische und strukturelle Beobachtungen um eine chemische Analyse ergänzen.

So konnte zum Beispiel die genaue Zusammensetzung der Gate-Pads ermittelt werden. Die Kombination dieser Techniken ermöglicht eine Bewertung der strukturellen und chemischen Eigenschaften auf mikroskopischer Ebene. Das ist besonders wichtig für die Optimierung von Halbleiterprozessen und die Verbesserung der Geräteleistung.



%------------------------------------------------------
%######################################################
%------------------------------------------------------

%\section{Formelzeichen}
%\input{Sections-ending/09-Formelzeichen}

%------------------------------------------------------
%######################################################
%------------------------------------------------------



%------------------------------------------------------
%######################################################
%------------------------------------------------------
\newpage
\section{Abbildungsverzeichnis}
\input{Sections-ending/11-Abbildungsverzeichnis}

%------------------------------------------------------
%######################################################
%------------------------------------------------------



%------------------------------------------------------
%######################################################
%------------------------------------------------------

\section{Literaturverzeichnis}
\input{Sections-ending/13-Literaturverzeichnis}

%------------------------------------------------------
%######################################################
%------------------------------------------------------

%------------------------------------------------------
%######################################################
%------------------------------------------------------


%------------------------------------------------------
%--------------------Vorlage-Testing-------------------
%------------------------------------------------------

%\section{Vorlage-Testing}
%\input{Vorlage/Figuren-Struktur}
%\input{Vorlage/Tabellen-Struktur}


\newpage

\end{document}