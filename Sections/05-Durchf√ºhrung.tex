
Die Versuchsdurchführung erfolgt im Gebäude C12, Raum 3.28. \\
Die Proben werden im Wasserbad umgekehrt von der untere Ebene eingelegt. So liegen die relevanten Schichtübergänge näher an der Fokuszone des Ultraschallwandlers. Dies verbessert die akustische Kopplung. Die Orientierung erleichtert außerdem die gezielte Analyse von Grenzflächen zwischen Halbleiter, Bondschicht und Substrat. \\
Nach dem Positionieren der Probe wird über die Gerätesoftware der C-Scan-Modus ausgewählt. Anschließend scannt der Ultraschallwandler die gewählte Fokusebene, und das Bild baut sich schrittweise auf dem Bildschirm auf. Über die integrierte Software des internen Rechners lassen sich die gemessenen Frequenzen ablesen. Ziel ist es, jene Frequenzen zu ermitteln, bei denen das Reflexionssignal am stärksten ausgeprägt ist.\\
Bei Probe 2 (siehe Abbildung 4) erfolgt die Messung in mehreren Fokusebenen, um die innere Struktur möglichst genau zu erfassen. Dazu wird die Höhe der Probe schrittweise angepasst, wobei jede Ebene erneut im C-Scan-Modus aufgenommen wird. Die einzelnen Scans zeigen unterschiedliche Materialgrenzen und mögliche Defekte. Durch den Vergleich dieser Ebenen lassen sich Aussagen über die Position und Tiefe der Fehlstellen treffen.