In dem letzten Versuch wurde die Anwendung des Rasterelektronenmikroskops (REM) dargestellt und erklärt. Allerdings tritt häufig das Problem auf, dass einige Proben in den inneren Schichten Schäden aufweisen, die mit dem REM nicht erkennbar sind, da es lediglich die Oberfläche analysieren kann.\\ An dieser Stelle kommt die SAM-Technologie (Scanning Acoustic Microscopy) zum Einsatz, da sie mithilfe von Ultraschallwellen die innere Struktur der Schichten sichtbar machen kann. In dieser Versuch wird dies nun durch praktische Anwendung bewiesen.