Der durchgeführte REM-Versuch untersuchte einen gesinterten Halbleiter mittels SE- und BSE-Detektor.\\
Als Leistungshalbleiter wird ein SIC-Mosfet benutzt, wo drei Strukturen analysiert wurden: \cite{key}
\begin{itemize}
    \item Gate Pad
    \item Sinterpad
    \item Source Pad
\end{itemize}
Zunächst die EDX Analyse einer gesinterten Sinterpaste in einem Al-Tiegel.\\
Für die Analyse wurde folgende Prüfgerät verwendet:
\begin{itemize}
    \item Rasterelektronenmikroskop der Firma Thermo Fisher Scientific
\end{itemize}
Für diese Analyse ist bei beiden Proben ein Rasterelektronenmikroskop zu verwenden. 

\subsection{Phenom XL G2 von Thermo Fisher Scientific}
Das Phenom XL G2 ist ein hochmoderner Rasterelektronenmikroskop, das für präzise Analyse von Strukturen und Zusammensetzung in der Materialwissenschaften Industrie entwickelt wurde:
Das REM besitzt die folgende Eigenschaften: \cite{2}
\begin{itemize}
    \item Automatisierte Abläufe:: Unterstützt die Automatisierung von Routineprozessen, wodurch manuelle und sich wiederholende Arbeitsschritte reduziert werden.
    \item Schneller Analysebeginn: Nach durchschnittlich 60 Sekunden liefert das Gerät erste Bilder.
    \item EDS-Funktion (Elementanalyse): Zur chemischen Analyse der Proben kann das System optional mit einem energie-dispersiven Röntgenspektrometer (EDS) ausgestattet werden, das die Elementzusammensetzung erfasst.
\end{itemize}

