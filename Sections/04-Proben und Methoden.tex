Zunächst wird die akustische Mikroskopie verwendet, um darstellen zu können, wie eine Zerstörung freie Untersuchung erfolgen kann, darüber hinaus werden nach Risse und Schaden untersucht und mögliche Darstellung bemerken zu können wie Luftblasen. Diese Untersuchung wird mit drei Proben durchgeführt.
Aufgabe bei den Proben ist auch die Ebenen zu wählen, die für eine solchen Untersuchung sinnvoll sind.  Potentielle
Defekte oder Fehler müssen dokumentiert und erläutert werden.


In diesem Experiment werden drei unterschiedliche Proben untersucht, um innere Strukturen und Defekte zerstörungsfrei zu erkennen. Die Messung erfolgt im Reflexionsmodus, wobei Ultraschallwellen über ein Wasserbad eingekoppelt und reflektierte Signale ausgewertet werden.
\begin{figure}[htbp]
    \centering
    \includegraphics[scale=0.11]{Bilder/ksiv8}
    \caption{KSI v300E Ultraschallmikroskop am Messplatz. Die Probe wird in einem Wasserbad positioniert und mittels eines piezoelektrischen Wandlers im Reflexionsmodus untersucht.}
    \vspace{0.2cm}
    \label{Abb.2: KSI v300E Ultraschallmikroskop am Messplatz. Die Probe wird in einem Wasserbad positioniert und mittels eines piezoelektrischen Wandlers im Reflexionsmodus untersucht. }
\end{figure} 
\vspace{0.2cm}

Die erste Probe ist eine keramische DCB-Leiterplatte mit gesinterten Halbleitern (siehe Abbildung 3). Sie dient zur Einführung in die SAM-Bedienung und zur Darstellung verschiedener Materialübergänge in mehreren Ebenen .
\begin{figure}[H]
    \centering
    \includegraphics[scale=0.12]{Bilder/probe1}
    \caption{DCB-Leiterplatte mit gesinterten Halbleitern. Die Probe dient zur Einarbeitung in die Bedienung des akustischen Mikroskops sowie zur Analyse von Materialübergängen.}
    
    \vspace{0.2cm}
    \label{Abb.3: DCB-Leiterplatte mit gesinterten Halbleitern. Die Probe dient zur Einarbeitung in die Bedienung des akustischen Mikroskops sowie zur Analyse von Materialübergängen. }
\end{figure} 
\vspace{0.2cm}

Die zweite Probe besteht aus einem Kupfer-Leadframe, der auf ein keramisches Substrat geschweißt wurde (siehe Abbildung 4). Hier liegt der Fokus auf der Analyse der Schweißverbindungen und der Untersuchung möglicher Defekte durch Wahl geeigneter Scanebenen .
\begin{figure}[H]
    \centering
    \includegraphics[scale=0.12]{Bilder/probe 2}
    \caption{Keramisches Substrat mit angeschweißtem Kupfer-Leadframe. Die Untersuchung konzentriert sich auf die Qualität der Schweißverbindungen und das Auffinden potenzieller Defekte.}
    \vspace{0.2cm}
    \label{Abb.4: Keramisches Substrat mit angeschweißtem Kupfer-Leadframe. Die Untersuchung konzentriert sich auf die Qualität der Schweißverbindungen und das Auffinden potenzieller Defekte. }
\end{figure} 
\vspace{0.2cm}

Die dritte Probe ist ein DoL-Leistungselektronikmodul ohne Bonddrähte (siehe Abbildung 5). Statt keramischer Isolation kommt eine organische Trägerfolie zum Einsatz. Die Ebenenwahl und Bewertung erfolgen selbstständig.
\begin{figure}[H]
    \centering
    \includegraphics[scale=0.13]{Bilder/probe3}
    \caption{DoL-Leistungselektronikmodul ohne Bonddrähte. Statt keramischer Isolation kommt eine organische Trägerfolie zum Einsatz. Ziel ist die eigenständige Auswahl und Bewertung geeigneter Fokusebenen.}
    \vspace{0.2cm}
    \label{Abb.5: DoL-Leistungselektronikmodul ohne Bonddrähte. Statt keramischer Isolation kommt eine organische Trägerfolie zum Einsatz. Ziel ist die eigenständige Auswahl und Bewertung geeigneter Fokusebenen. }
\end{figure} 
\vspace{0.2cm}
Alle Proben werden im Wasserbecken positioniert, wobei auf die Entfernung von Luftblasen geachtet wird. Die Steuerung erfolgt über den integrierten PC, und nach jeder Messung wird das Gerät in die Home-Position zurückgefahren.



Die Untersuchung erfolgt mit dem Scanning Acoustic Microscope KSI V8 \cite{hennig2025}.

Das System bietet mehrere Scanmodi, die eine detaillierte Untersuchung unterschiedlicher Strukturen ermöglichen. Im sogenannten C-Scan-Modus wird eine zweidimensionale Darstellung einer definierten Fokusebene erzeugt, die sich besonders zur Erkennung von Hohlräumen und Rissen innerhalb einzelner Materialschichten eignet. Ergänzend dazu erlaubt der B-Scan die Darstellung eines Tiefenprofils, während der 3D-Scan durch die Kombination mehrerer Ebenen eine volumetrische Abbildung des Probenaufbaus liefert.




