Zunächst wird die akustische Mikroskopie verwendet, um darstellen zu können, wie eine Zerstörung freie Darstellung erfolgen kann, darüber hinaus werden werden nach Risse und irgendwelche Schaden untersucht und mögliche Darstellung bemerken zu können wie Luftblasen. Diese Untersuchung wird mit drei Proben durchgeführt.
Auffabe bei den proben ist auch die Ebenen zu wählen, die für eine solchen Untersuchung sinnvoll sind.  Potentielle
Defekte oder Fehler müssen aufgenommen und erklärt werden.


In diesem Experiment werden drei unterschiedliche Proben mit dem KSI V8 SAM untersucht, um innere Strukturen und Defekte zerstörungsfrei zu erkennen. Die Messung erfolgt im Reflexionsmodus, wobei Ultraschallwellen über ein Wasserbad eingekoppelt und reflektierte Signale ausgewertet werden.

Die erste Probe ist eine keramische DCB-Leiterplatte mit gesinterten Halbleitern, teils mit Bonddrähten kontaktiert (siehe Abbildung 1). Sie dient zur Einführung in die SAM-Bedienung und zur Darstellung verschiedener Materialübergänge in mehreren Ebenen.

Die zweite Probe besteht aus einem Kupfer-Leadframe, der auf ein keramisches Substrat geschweißt wurde. Hier liegt der Fokus auf der Analyse der Schweißverbindungen und der Untersuchung möglicher Defekte durch Wahl geeigneter Scanebenen.

Die dritte Probe ist ein DoL-Leistungselektronikmodul ohne Bonddrähte. Statt keramischer Isolation kommt eine organische Trägerfolie zum Einsatz. Die Ebenenwahl und Bewertung erfolgen selbstständig.

Alle Proben werden im Wasserbecken positioniert, wobei auf die Entfernung von Luftblasen geachtet wird. Die Steuerung erfolgt über den integrierten PC, und nach jeder Messung wird das Gerät in die Home-Position zurückgefahren.
