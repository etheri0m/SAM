In diesem Versuch wird die akustische Mikroskopie (SAM) eingesetzt, um zu demonstrieren, wie eine zerstörungsfreie bildgebende Untersuchung innerer Strukturen in Materialien durchgeführt werden kann. Dabei sollen insbesondere potenzielle Defekte wie Risse, Delaminationen, Lufteinschlüsse oder andere strukturelle Schäden identifiziert und analysiert werden.
Die Untersuchung erfolgt an drei verschiedenen Proben, bei denen gezielt geeignete Untersuchungsebenen gewählt werden müssen, um relevante Fehlstellen sichtbar zu machen. Ziel ist es, auffällige Bereiche zu dokumentieren, die jeweiligen Fehlerarten zu identifizieren und deren Ursache sowie Auswirkungen auf die Materialstruktur zu erläutern.