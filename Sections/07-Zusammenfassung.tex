
Im Rahmen des Versuchs zur akustischen Mikroskopie wurden drei unterschiedliche Proben untersucht: ein DCB-Modul mit gesinterten Halbleitern, ein geschweißter Kupfer-Leadframe auf Keramik sowie ein DoL-Leistungselektronikmodul mit organischer Trägerfolie. Ziel ist es, mithilfe des C-Scan-Modus im Reflexionsverfahren die inneren Materialschichten und Grenzflächen zerstörungsfrei zu analysieren und visuell darzustellen.

Die Untersuchungen erfolgten in mehreren Fokusebenen, insbesondere bei der zweiten Probe, um strukturbedingte Unterschiede sichtbar zu machen. Das verwendete KSI V8 SAM ermöglichte dabei hochauflösende Aufnahmen, aus denen potenzielle Defekte, Delaminationen und Unregelmäßigkeiten abgeleitet werden konnten.\\
Die Ergebnisse zeigen, dass das DoL-Modul (Probe 3) weitgehend homogen aufgebaut ist und keine auffälligen strukturellen Defekte aufweist. Lediglich einige mögliche Oberflächenverunreinigungen oder Lufteinschlüsse sind erkennbar. Auffällig ist die klar lesbare Gravur „FH-KIEL“, was auf eine gute Oberflächenreflexion und geringe Dämpfung hindeutet.

Bei der ersten Probe (DCB-Modul) konnten keine Risse oder Delaminationen festgestellt werden. Die Draht-Bonds erscheinen vollständig intakt, und die Kontaktierungen zwischen den Kupferflächen sind klar sichtbar. Die zweite Probe (Leadframe auf Keramik) offenbarte bei der schichtweisen Untersuchung einzelne Unregelmäßigkeiten in tieferen Ebenen. Diese deuten auf Schwachstellen in der Materialanbindung hin, insbesondere entlang der unteren Bondschicht.