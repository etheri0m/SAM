Im Rahmen des Versuchs zur akustischen Mikroskopie wurden drei unterschiedliche Proben untersucht: ein DCB-Modul mit gesinterten Halbleiterchips, ein Kupfer-Leadframe, der mittels Ultraschallschweißung auf ein Keramiksubstrat aufgebracht wurde, sowie ein DoL-Leistungsmodul mit einer organischen Trägerfolie. Ziel der Untersuchung war es, mithilfe des C-Scan-Modus im Reflexionsverfahren (Puls-Echo-Technik) die inneren Schichtstrukturen und Grenzflächen der Materialien zerstörungsfrei darzustellen und zu analysieren. Dabei macht man sich zunutze, dass Ultraschallwellen an Grenzflächen mit unterschiedlicher akustischer Impedanz reflektiert werden, wodurch innere Materialstrukturen sichtbar gemacht werden können \cite{pvateplaSAM, wikiSAM2024}.

Zur Erfassung der inneren Defekte und Heterogenitäten wurden die Proben in mehreren Fokusebenen untersucht also bei variierender Tiefeneinstellung im C-Scan \cite{hennig2025}. Besonders bei Probe 2 wurde dieses Verfahren genutzt, um schichtabhängige Unterschiede im Reflexionsverhalten aufzudecken. Für eine zuverlässige Kopplung der Schallwellen an das Probenmaterial wurde Wasser als Koppelmedium eingesetzt, da Luft aufgrund ihrer geringen akustischen Impedanz kaum Ultraschall überträgt \cite{pvateplaSAM}.

Das verwendete akustische Mikroskop vom Typ KSI V8 ermöglichte hochauflösende, kontrastreiche Aufnahmen der Probenquerschnitte. Die dabei gewonnenen Bilddaten lieferten eine klare Visualisierung möglicher Defekte wie Delaminationen, Inhomogenitäten oder Hohlräume \cite{hennig2025}. Besonders beim DoL-Modul (Probe 3) zeigten sich einzelne kleine Flecken, vermutlich Oberflächenverunreinigungen oder eingeschlossene Luftblasen, konnten detektiert werden. Solche Inhomogenitäten erzeugen aufgrund des stark abweichenden akustischen Widerstandes besonders helle Reflexionssignale, häufig begleitet von einer Phasenumkehr im Echo \cite{wikiSAM2024}.

Besonders hervorzuheben ist die klare Darstellung des Gravurschriftzugs „FH-KIEL“ im C-Scan. Dies deutet auf eine gleichmäßige und intakte Oberflächenstruktur hin, die zu einer starken Reflexion und nur geringer Signalabschwächung führt. In Probe 1 ( DCB-Modul) ließen sich ebenfalls keine strukturellen Schwächen wie Delaminationen oder Risse nachweisen. Die Bondverbindungen erscheinen vollständig intakt, ebenso die metallischen Kontaktflächen, was auf eine homogene Materialanbindung ohne Hohlräume oder Ablösungen schließen lässt.

Im Gegensatz dazu offenbarte Probe 2, also der Kupfer-Leadframe auf Keramik, bei schrittweiser Tiefenfokussierung mehrere lokal begrenzte Auffälligkeiten in den tiefer liegenden Schichten \cite{hennig2025}. Im C-Scan erscheinen diese als deutlich abgegrenzte Zonen mit erhöhter Reflexionsintensität, was auf Schwachstellen in der strukturellen Verbindung hindeutet \cite{pvateplaSAM}. Besonders an der unteren Grenzfläche zwischen Bondschicht und Keramiksubstrat wurden solche Anomalien sichtbar. Es ist anzunehmen, dass es sich hierbei um partielle Delaminationen oder Lufteinschlüsse handelt. Diese verursachen durch den abrupten Wechsel der akustischen Impedanz eine nahezu vollständige Reflexion des Ultraschalls, häufig inklusive Phasenumkehr \cite{wikiSAM2024}.

Durch die gezielte Fokussierung auf diese tieferliegenden Ebenen und das präzise Setzen eines Laufzeitfensters im C-Scan konnte die exakte Tiefenlage dieser Defekte lokalisiert und in der Bildauswertung eindeutig zugeordnet werden \cite{hennig2025}.