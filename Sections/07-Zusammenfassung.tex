Die SEM- und EDX-Analysen zeigen Unterschiede zwischen den verschiedenen Bereichen. Das Gate-Pad und das Source-Pad bestehen hauptsächlich aus Gold, Palladium und etwas Nickel, was typisch für leitfähige Kontakte ist.\\ 
Die Sinterpads weisen je nach Messstelle unterschiedliche Materialverteilungen und Sinterzustände auf. Besonders das linke Sinterpad fällt durch seinen hohen Silberanteil auf.\\\\
Die Bilder des Sinterpads vom SE- und BSE-Detektor sehen sehr ähnlich aus. 
Der auffälligste Unterschied liegt im Kontrast. 
Das SE-Bild ist etwas dunkler und zeigt stärkere Kontraste. 
Es hat eine Informationstiefe von etwa 5nm. Der BSE-Detektor hingegen kann bis zu 100nm tief eindringen. 
Das ist besonders hilfreich bei der Analyse der chemischen Zusammensetzung. Durch den Materialkontrast erscheinen Elemente mit höherer Ordnungszahl heller. \\ \\
Der Aluminiumtiegel stellt während der Messung eine gewisse Schwierigkeit dar. 
Zum Beispiel wurde beim Messen des rechten Sinterpads eine auffällige Anomalie festgestellt: Die EDX-Messung zeigte einen ungewöhnlich hohen Aluminiumanteil. Der Grund dafür war, dass der Messpunkt zu nah am Aluminiumtiegel lag, sodass die detektierten Elektronen tatsächlich vom Tiegel und nicht vom Pad selbst stammten.\\
Das Hauptproblem besteht darin, dass die gemessenen Elektronen durch die Elektronen gestört werden, die vom Aluminium-Tiegel emittiert werden. Diese Wechselwirkung beeinflusst die Messergebnisse erheblich und erschwert die Identifizierung der Sinterpaste, solange sie nicht vom umgebenden Aluminium isoliert ist.\\
Insgesamt liefern die Aufnahmen wichtige topografische und analytische Informationen über den Aufbau und die Materialqualität im Chip.