Dieses Experiment hat erfolgreich gezeigt, wie ein Rasterelektronenmikroskop (REM) funktioniert und wie die verschiedenen Detektoren eingesetzt werden. Die Untersuchung des SiC-MOSFETs und der gesinterten Paste lieferte dabei wichtige Einblicke in die Materialeigenschaften und die Leistungsfähigkeit des REM.

Die Ergebnisse belegen, dass die Kombination aus SE-Detektor, BSE-Detektor und EDX-Analyse gut geeignet ist, um Oberflächen zu untersuchen und Materialien zu identifizieren. Durch die EDX-Analyse lassen sich topografische und strukturelle Beobachtungen um eine chemische Analyse ergänzen.

So konnte zum Beispiel die genaue Zusammensetzung der Gate-Pads ermittelt werden. Die Kombination dieser Techniken ermöglicht eine Bewertung der strukturellen und chemischen Eigenschaften auf mikroskopischer Ebene. Das ist besonders wichtig für die Optimierung von Halbleiterprozessen und die Verbesserung der Geräteleistung.
